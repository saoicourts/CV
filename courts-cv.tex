%%%%%%%%%%%%%%%%%%%%%%%%%%%%%%%%%%%%%%%%%
% Medium Length Graduate Curriculum Vitae
% LaTeX Template
% Version 1.1 (9/12/12)
%
% This template has been downloaded from:
% http://www.LaTeXTemplates.com
%
% Original author:
% Rensselaer Polytechnic Institute (http://www.rpi.edu/dept/arc/training/latex/resumes/)
%
% Important note:
% This template requires the res.cls file to be in the same directory as the
% .tex file. The res.cls file provides the resume style used for structuring the
% document.
%
%%%%%%%%%%%%%%%%%%%%%%%%%%%%%%%%%%%%%%%%%

%----------------------------------------------------------------------------------------
%	PACKAGES AND OTHER DOCUMENT CONFIGURATIONS
%----------------------------------------------------------------------------------------

\documentclass[margin]{res} % Use the res.cls style, the font size can be changed to 11pt or 12pt here

\usepackage{helvet} % Default font is the helvetica postscript font
%\usepackage{newcent} % To change the default font to the new century schoolbook postscript font uncomment this line and comment the one above
\usepackage{amsmath}

\usepackage[utf8]{inputenc}
\usepackage[russian,english]{babel}

\setlength{\textwidth}{5.1in} % Text width of the document
\usepackage{standalone}

\begin{document}

%----------------------------------------------------------------------------------------
%	NAME AND ADDRESS SECTION
%----------------------------------------------------------------------------------------

\moveleft.5\hoffset\centerline{\large\bf Nico Courts} % Your name at the top
 
\moveleft\hoffset\vbox{\hrule width\resumewidth height 1pt}\smallskip % Horizontal line after name; adjust line thickness by changing the '1pt'

\moveleft.5\hoffset\centerline{Ph.D. Candidate}
\moveleft.5\hoffset\centerline{University of Washington Department of Mathematics}
\moveleft.5\hoffset\centerline{nico@nicocourts.com\qquad ncourts@uw.edu\qquad github.com/NicoCourts}

%----------------------------------------------------------------------------------------

\begin{resume}


%----------------------------------------------------------------------------------------
%	EDUCATION SECTION
%----------------------------------------------------------------------------------------

\section{EDUCATION}

{\bf Ph.D.,} Mathematics\hfill June 2022 (expected) \\
University of Washington, Seattle, WA

{\bf Master of Science,} Mathematics\hfill March 2020 \\
University of Washington, Seattle, WA \\
{\it Thesis Topic:} Schur Duality and Strict Polynomial Functors (paper on github)

{\bf Bachelor of Science,} Mathematics\hfill May 2016 \\
{\sl Magna Cum Laude}, Phi Beta Kappa, Dean's List, Departmental Honors \\
University of Southern California, Los Angeles, CA 

{\bf Budapest Semesters in Mathematics}\hfill Fall 2015 \\
Algebraic Topology, Conjecture \& Proof, Cryptography, Differential Geometry  \\
Budapest, Hungary 

{\bf Associate of Science,} Mathematics\hfill June 2013\\
Key of Knowledge, Dean's list, Honors Program \\
Citrus College, Glendora, CA 
 
%----------------------------------------------------------------------------------------
%	RESEARCH EXPERIENCE SECTION
%----------------------------------------------------------------------------------------

\section{RESEARCH INTERESTS}

\textbf{Representation Theory}\\
I am a student of Prof. Julia Pevtsova studying the representation theory of certain Hopf Algebras
called bosonized quantum complete intersections. As their names suggest, they are close noncommutative
relatives of complete intersection rings and have a similar flavor to their classical cousins. I am interested 
in particular in studying different notions of support for these algebras in the hopes of establishing
an incarnation of the beloved tensor product property of support.

\textbf{Machine Learning}\\
I am also an intern with Pacific Northwest National Laboratory under the instruction of Dr. Henry Kvinge. 
I am particularly interested in using my familiarity with concepts from algebra, category theory, and geometry
to enhance the current understanding of deep learning models with a focus towards providing more robust interpretability.
Thus far my experience has included few-shot learning and generative models (specifically invertible NNs for normalizing flows), both of 
which have resulted in papers.

\section{INTERNSHIPS}
\textbf{NSIP PhD Intern}\hfill June 2020 -- Present\\
Pacific Northwest National Labs, Seattle, WA

A full-time summer internship spanning 2020 and 2021 including part-time work over the intervening school year. Our first 10 week project was focused on machine learning algorithms that could incorporate user input at inference time to hone results. Few-shot learning was the natural place to focus our energy and we developed and implemented \textit{Fuzzy Simplicial Networks}, a metric-based model that we showed was able to make strong inferences on transfer tasks without fine-tuning by exploiting the geometry of feature space.

During the 2020/2021 year, some of my work on out-of-support detection for few-shot learning culminated in a paper along with my collaborators, although my contribution to this work was limited due to my scholarly responsibilities.

Finally, in summer 2021 we joined a team of data scientists who were interested in helping material scientists at the lab understand and refine their fabrication methods. Inspired by the idea of fiber bundles from topology, we developed a new generative architecture based on invertible neural nets like RNVP and GIN that used local trivializations to achieve superior density estimations than several GANs. In addition, our architecture was able to sample from the ``fiber'' over a point (all the points in input space that map to a particular point) much more effectively than simply conditioning on the output.

\section{PREPRINTS}

\textbf{Bundle Networks: Fiber Bundles, Local Trivializations, and a Generative Approach to Exploring Many-to-one Maps}\\
Nico Courts and Henry Kvinge\\
Submitted to ICLR 2022\\
Preprint available at https://arxiv.org/abs/2110.06983

\textbf{One Representation to Rule Them All: Identifying Out-of-Support Examples in Few-shot Learning with Generic Representations}\\
Henry Kvinge, Scott Howland, Nico Courts, Lauren A. Phillips, John Buckheit, Zachary New, Elliott Skomski, Jung H. Lee, Sandeep Tiwari, Jessica Hibler, Courtney D. Corley, Nathan O. Hodas\\
Preprint available at https://arxiv.org/abs/2106.01423

\section{PUBLICATIONS}

\textbf{Fuzzy Simplicial Networks: A Topology-Inspired Model to Improve Task Generalization in Few-shot Learning}\\
Henry Kvinge, Zachary New, Nico Courts, Jung H. Lee, Lauren A. Phillips, Courtney D. Corley, Aaron Tuor, Andrew Avila, Nathan O. Hodas\\
\textit{AAAI Workshop on Meta-Learning and MetaDL Challenge}, PMLR 140:77-89, 2021. Available at https://proceedings.mlr.press/v140/kvinge21a.html


%----------------------------------------------------------------------------------------
%	PROFESSIONAL EXPERIENCE SECTION
%----------------------------------------------------------------------------------------
 
\section{TEACHING EXPERIENCE}

{\bf Graduate Teaching Assistant} \hfill Autumn 2016 -- Present \\
University of Washington, Seattle, WA

\begin{itemize} \itemsep -1pt % Reduce space between items
\item\textbf{As an instructor:} 
\begin{itemize}
	\item Math 124 -- Calculus I (Su 2018)
	\item Math 208 -- Matrix Algebra (Sp 2019)\vspace{5pt}

\end{itemize}
\item \textbf{As a teaching assistant:} 
\begin{itemize}
	\item Math 120 -- Precalculus (Au 2017)
	\item Math 124 -- Calculus I (Wi 2017, Wi 2019, Sp 2020)
	\item Math 125 -- Calculus II (Au 2016, Sp 2017, Wi 2021)
	\item Math 126 -- Calculus III (Su 2017, Wi 2018, Sp 2018, Au 2020)
	\item Math 208 -- Matrix Algebra (Au 2021)
	\item Math 327 -- Introductory Real Analysis (Su 2019)
	\item Math 381 -- Discrete Mathematical Modeling (Au 2018)
	\item Math 403 -- Group Theory (Wi 2020)
\end{itemize}
\end{itemize}
 
\pagebreak
{\bf Lead Teaching Assistant and Instructor} \hfill Summer 2016 \\
SCS Noonan Scholars (previously South Central Scholars), Los Angeles, CA
\begin{itemize} \itemsep -1pt
\item Independently developed and delivered  approximately 50 hours of instruction and five exams to gifted university-bound students in calculus 2 and 3.
\item Total of 100 contact hours, including daily supervised worksheet sessions.
\item Took the initiative to deliver weekly lectures in higher mathematics (number theory, knot theory, differential equations, etc.) along with entry-level problems that allowed students to get a sense of the “flavor” of these fields.
\end{itemize} 

{\bf Various Teaching and Mentorship Positions} \hfill Spring 2012 -- Summer 2013 \\
Citrus College, Glendora, CA
\begin{itemize} \itemsep -1pt
\item {\bf PAGE Program Tutor} Assisted a licensed teacher in the education of a class of middle school children intended to reinforce the previous year’s learning and to prevent “backsliding”. Personally instructed a small group of students who were prepared to learn more advanced topics in intermediate algebra.
\item {\bf SIGMA Mentor} Took on a small group of students each semester utilizing a holistic approach to education – supplementing standard tutoring with more in-depth educational guidance and planning.
\item {\bf Math Tutor} Instructed students in the fast-paced Math Success Center where I provided homework help in all math classes through linear algebra and differential equations.
\end{itemize} 

\section{LEADERSHIP\\ \& SERVICE}

{\bf Washington Directed Reading Program}\\
{\sl Co-organizer}\\
Autumn 2019 - Spring 2021
\begin{itemize}\itemsep -1pt
	\item Helped run the WDRP along with two other students. Each quarter we read applications, made acceptance and pairing decisions, held several events for enrichment and networking, and handled the administrivia required to keep projects running smoothly.
\end{itemize}
{\sl Mentor}
\begin{itemize}\itemsep -1pt
    \item Introduced a student to group representation theory based on Artin's \textit{Algebra} as well as Fulton \& Harris' \textit{Representation Theory} (Autumn 2021).
	\item Supervised an undergraduate student in a reading course based around Rebecca Weber's book {\sl Computability Theory} (Autumn 2018).
\end{itemize}

\textbf{Institute for the Quantitative Study of Inclusion, Diversity, and Equity (QSIDE) -- Datathon4Good}\\
\textit{Team Leader}\\
October 2021
\begin{itemize}
    \item Served alongside two faculty members to help guide a team of 13 students and faculty from across the country in applying data analysis and visualization tools in service of social justice.
    \item Brought in my experience in math, computer science, and data visualization to help guide the team in their analysis of two datasets related to incarceration and criminal justice.
\end{itemize}

\pagebreak
\section{LEADERSHIP\\ \& SERVICE (CONT.)}
{\bf Graduate Student Representative}\\
University of Washington, Seattle\\
Summer 2019 - Spring 2020
\begin{itemize} \itemsep -1pt
	\item Planned and organized a variety of events and lectures for the graduate students as well as the department at large.
    \item Served as an advocate for the graduate students in several capacities. 
	\item Worked on promoting better communication between the students and faculty.
	\item Empowered students to make changes to the department while promoting respect for the wishes of the faculty and administration.
\end{itemize}

\textbf{Departmental Union Steward}\\
University of Washington, Seattle\\
Winter 2020 - Present

{\bf Math Hour Olympiad}\\
Volunteer Judge \\
University of Washington, Seattle\\
Spring 2018

{\bf Math Day}\\
Volunteer\\
University of Washington, Seattle\\
2017 and 2018

\section{SEMINARS ORGANIZED}
\textbf{Pacific Northwest Seminar on Topology, Algebra, and Geometry in Data Science (TAG-DS)}\\
\textit{Co-Organizer}\\
Joint seminar between UW and PNNL platforming speakers who work at the intersection of pure math and data science.\\
Autumn 2021 -- present

\textbf{Departmental Current Topics Seminar}\\
\textit{Organizer}\\
Internal seminar for professors to talk about their research with the hopes of joining students with potential advisors.\\
Autumn 2019


\section{TALKS GIVEN}

\textbf{Fuzzy Simplicial Networks: A Topology-Inspired Model to Improve Task Generalization in Few-shot Learning}\\
\textit{5th Workshop on Geometry and Machine Learning}\\
International Symposium on Computational Geometry (online)\\
June 8, 2021

\textbf{Geometry of the Loss Landscape} (presenter/discussion leader)\\
\textit{PNNL Math of Machine Learning Reading Group}\\
Pacific Northwest National Laboratory (online)\\
March 24, 2021

\textbf{Schur Algebras \& Duality}\\
\textit{Special Colloquium Series for Mathematical Sciences}\\
Georgia Southern University (online)\\
November 20, 2020

\section{\bf EVENTS ATTENDED}

\textbf{International Conference on Machine Learning}\\
Online\\
June 2021

\textbf{International Conference on the Representation of Algebras}\\
Online\\
November 2020

{\bf Conference on Lie and Jordan Algebras and their Representations}\\
Sichuan University\\
Chengdu, Sichuan Province, P.R. China\\
January 2020

{\bf Triangulated Categories in Representation Theory and Geometry}\\
University of Sydney\\
Sydney, NSW, Australia\\
June 2019

{\bf MSRI Summer School}\\
{\sl The Mathematics of Machine Learning}\\
University of Washington\\
Seattle, WA\\
July 29 - August 9, 2019

{\bf ABC Workshop}\\
{\sl Geometric and Cohomological Methods in Algebra}\\
University of Washington\\
Seattle, WA\\
November 11, 2018

{\bf Joint Mathematical Meetings}\\
Seattle, WA\\
January 2016

%----------------------------------------------------------------------------------------
%	SKILLS SECTION
%----------------------------------------------------------------------------------------

\section{SKILLS \& HOBBIES} 

{\bf Languages:}
\begin{itemize} \itemsep -2pt
	\item {\sl English} -- This is my native language.
	\item {\sl German} -- Ich kann ziemlich gut Deutsch sprechen, lesen, und verstehen! (proficient)
	\item {\sl Russian} -- \foreignlanguage{russian}{Я немного понимаю по-русски.} (beginner)
	\item {\sl Programming} -- {\bf Go}, Haskell, Java, {\bf Javascript, \LaTeX}, PHP, {\bf Python}.
\end{itemize}
{\bf Computer Skills:} Web/application development, server administration, Sage, Windows, Linux, FreeBSD.

{\bf Life Skills:} Critical thinking, abstract reasoning, communication, objectivity, empathy.

{\bf Hobbies:} Hiking, jogging, rollerskating, and appreciating the wonders of the pacific northwest.



\end{resume}
\end{document}